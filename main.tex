\documentclass[aspectratio=169]{beamer}
\usepackage{polski}
\usepackage{biblatex}
\usetheme[lang=pl,hr=true]{NewPwr}
\author{Filip Godlewski, 290315}
\title{Technologie informacyjne}
\subtitle{Prezentacja z labolatorium Technologii informacyjnych}
\institute{Politechnika Wrocławska}
\date{\today}

\begin{document}

\begin{frame} % Slajd tytułowy
 \maketitle
\end{frame}

\begin{frame}
 \frametitle{Zaliczenie zadania}
 https://github.com/godlewskifilip/prezentacja
\end{frame}

\begin{frame}
 \frametitle{Plan prezentacji}
 \begin{itemize}
  \item
        Rozdział 1: Wprowadzenie do technologii informacyjnych,
  \item
        Rozdział 2: Kluczowe obszary zastosowań technologii informacyjnych,
  \item
        Rozdział 3: Etapy rozwoju technologii informacyjnych,
  \item
        Rozdział 4: Znaczenie technologii informacyjnych w społeczeństwie,
  \item
        Rozdział 5: Przyszłość technologii informacyjnych:
  \item
        Rozdział 5.1: Rozwój technologii,
  \item
        Rozdział 5.2: Wyzwania technologiczne przyszłości,
  \item
        Rozdział 6: Wnioski,
  \item
        Podsumowanie,
  \item
        Spis materiałów.
 \end{itemize}
\end{frame}

\begin{frame}
 \frametitle{Rozdział 1: Wprowadzenie do technologii informacyjnych}
 Technologie informacyjne to zbiór narzędzi, metod i procesów, które umożliwiają przetwarzanie, gromadzenie, przesyłanie i wykorzystywanie informacji w sposób efektywny. Ich rozwój w ostatnich dekadach diametralnie zmienił sposób, w jaki funkcjonują społeczeństwa, gospodarki oraz instytucje publiczne. Współczesny świat praktycznie nie może istnieć bez technologii informacyjnych, ponieważ są one podstawą niemal każdego sektora życia. Rozwój technologii informacyjnych nie tylko usprawnia komunikację i dostęp do danych, ale także kształtuje nowe modele biznesowe, edukacyjne oraz administracyjne. W tym obszarze powstają innowacyjne rozwiązania, które wspierają inteligentne miasta, zrównoważony rozwój oraz poprawę jakości życia w społeczeństwach.
\end{frame}

\begin{frame}
 \frametitle{Rozdział 2: Kluczowe obszary zastosowań technologii informacyjnych}
    Technologie informacyjne obejmują wiele dziedzin, z których każda pełni
    inną rolę w procesie cyfrowej transformacji. Najważniejsze z nich to:
    \begin{enumerate}
        \def\labelenumi{\arabic{enumi}.}
        \item
          Infrastruktura IT -- sprzęt komputerowy, serwery, sieci i urządzenia
          mobilne, które tworzą fundament systemów informatycznych. Obejmuje
          także centra danych, serwery w chmurze, urządzenia IoT i sprzęt do
          analizy danych w czasie rzeczywistym.
        \item
          Oprogramowanie -- systemy operacyjne, aplikacje użytkowe, systemy
          zarządzania bazami danych, które pozwalają użytkownikom wykonywać
          konkretne zadania i analizować dane. Oprogramowanie staje się coraz
          bardziej inteligentne dzięki integracji algorytmów uczenia maszynowego
          i sztucznej inteligencji.
        \item
          Usługi sieciowe i chmurowe -- umożliwiają przechowywanie, przesyłanie
          i przetwarzanie danych w środowisku online, oferując skalowalność i
          elastyczność. Popularne modele to Software as a Service (SaaS),
          Platform as a Service (PaaS) i Infrastructure as a Service (IaaS).
    \end{enumerate}
\end{frame}

\begin{frame}
 \frametitle{Rozdział 2: Kluczowe obszary zastosowań technologii informacyjnych}
    \begin{enumerate}
    \setcounter{enumi}{3}
        \def\labelenumi{\arabic{enumi}.}
        \item
          Cyberbezpieczeństwo -- obejmuje metody ochrony danych i systemów przed
          atakami, włamaniami, a także przed nieautoryzowanym dostępem. W tym
          obszarze szczególnie istotne są technologie szyfrowania,
          uwierzytelniania wielopoziomowego oraz monitoring zagrożeń.
        \item
          Analiza danych i sztuczna inteligencja -- umożliwia przetwarzanie
          ogromnych ilości danych, wykrywanie wzorców, przewidywanie trendów i
          automatyzację procesów. Narzędzia analityczne wspierają decyzje
          biznesowe, prognozowanie rynków oraz personalizację usług.
        \item
          Rozwój interfejsów użytkownika -- projektowanie intuicyjnych i
          dostępnych systemów, które ułatwiają interakcję człowieka z
          technologią. Dotyczy to zarówno aplikacji mobilnych, jak i
          rozbudowanych systemów korporacyjnych.
    \end{enumerate}
\end{frame}

\begin{frame}
 \frametitle{Rozdział 2: Kluczowe obszary zastosowań technologii informacyjnych}
    \begin{enumerate}
    \setcounter{enumi}{6}
        \def\labelenumi{\arabic{enumi}.}
        \item
          Telekomunikacja i łączność bezprzewodowa -- podstawy działania sieci
          komórkowych, Wi-Fi, sieci 5G oraz przyszłych technologii 6G, które
          umożliwiają natychmiastowy dostęp do informacji i komunikację w czasie
          rzeczywistym.
    \end{enumerate}
    
    Każda z tych dziedzin jest rozwijana równolegle na całym świecie, co
    prowadzi do powstawania innowacyjnych rozwiązań i zwiększa złożoność
    całego ekosystemu technologii informacyjnych.

\end{frame}

\begin{frame}
 \frametitle{Rozdział 3: Etapy rozwoju technologii informacyjnych}
Rozwój technologii informacyjnych można podzielić na kilka kluczowych
etapów, z których każdy wnosi nowe możliwości i wyzwania:

\begin{itemize}
\item
  Era mechaniczna -- pierwsze urządzenia liczące, takie jak abakus,
  mechaniczne kalkulatory i maszyny Babbage'a. Był to początek
  automatyzacji obliczeń i wprowadzenie systematycznego podejścia do
  przetwarzania danych.
\item
  Era elektroniczna -- pojawienie się komputerów opartych na układach
  scalonych i tranzystorach. Rozwój systemów operacyjnych umożliwił
  efektywne zarządzanie zasobami sprzętowymi oraz tworzenie aplikacji do
  obliczeń naukowych i biznesowych.
\item
  Era sieciowa -- rozwój Internetu, sieci komputerowych i globalnej
  komunikacji, które zrewolucjonizowały wymianę informacji na świecie. W
  tym okresie powstały pierwsze przeglądarki, poczta elektroniczna i
  protokoły sieciowe.
\end{itemize}
\end{frame}

\begin{frame}
 \frametitle{Rozdział 3: Etapy rozwoju technologii informacyjnych}
\begin{itemize}
\item
  Era mobilna -- dominacja urządzeń przenośnych, smartfonów, tabletów i
  technologii bezprzewodowych, umożliwiających dostęp do informacji w
  czasie rzeczywistym oraz rozwój aplikacji mobilnych i platform
  społecznościowych.
\item
  Era sztucznej inteligencji -- integracja systemów uczących się i
  automatyzacja procesów decyzyjnych, rozwój chatbotów, asystentów
  głosowych oraz systemów analitycznych w przedsiębiorstwach. AI wspiera
  również procesy predykcyjne, automatyczne diagnozy medyczne i
  zarządzanie ruchem miejskim.
\item
  Era Internetu Rzeczy (IoT) -- połączenie urządzeń codziennego użytku z
  Internetem, umożliwiające zbieranie danych, inteligentne zarządzanie
  środowiskiem fizycznym, monitorowanie infrastruktury miejskiej oraz
  automatyzację gospodarstw domowych.
\end{itemize}
\end{frame}

\begin{frame}
 \frametitle{Rozdział 3: Etapy rozwoju technologii informacyjnych}
\begin{itemize}
\item
  Era komputerów kwantowych i zaawansowanych algorytmów -- wprowadzenie
  nowych paradygmatów obliczeniowych, które znacząco zwiększają moc
  przetwarzania danych i rozwiązywania złożonych problemów, takich jak
  modelowanie molekularne, optymalizacja logistyczna czy kryptografia
  postkwantowa. Każdy z tych etapów charakteryzował się nie tylko
  wzrostem możliwości technicznych, ale także koniecznością adaptacji
  społecznej, edukacyjnej i prawnej. W miarę rozwoju technologii
  powstają też nowe modele regulacyjne, standardy i normy
  bezpieczeństwa, które mają chronić użytkowników i organizacje.
\end{itemize}
\end{frame}

\begin{frame}
\begin{table}
\centering
\scriptsize
\resizebox{\textwidth}{!}{%
\begin{tabular}{|p{2.2cm}|p{2.8cm}|p{2.8cm}|p{2.6cm}|p{3cm}|}
\hline
\textbf{Technologia} &
\textbf{Zastosowanie} &
\textbf{Zalety} &
\textbf{Wyzwania} &
\textbf{Przykłady praktyczne} \\ \hline

Chmura sieciowa &
Przechowywanie i przetwarzanie danych &
Skalowalność, dostępność zdalna &
Bezpieczeństwo, koszty &
AWS, Google Cloud, Microsoft Azure \\ \hline

Sztuczna inteligencja &
Analiza danych, automatyzacja procesów &
Predykcja trendów, automatyzacja &
Etyka, wymagana jakość danych &
Chatboty, systemy rekomendacyjne \\ \hline

Internet rzeczy &
Smart home, inteligentne miasta &
Zdalne monitorowanie, oszczędzanie energii &
Prywatność, interoperacyjność &
Smart home devices, systemy monitoringu miasta \\ \hline

Blockchain &
Kryptowaluty, rejestry transakcji &
Niezmienność danych, transparentność &
Skalowalność, zużycie energii &
Bitcoin, Ethereum, rejestry łańcuchowe \\ \hline

Telemedycyna &
Zdalna opieka medyczna &
Wygoda pacjenta, monitorowanie stanu zdrowia &
Prywatność danych, wymogi regulacyjne &
Konsultacje online, monitorowanie \\ \hline
\end{tabular}%
}

\caption{Przykładowe zastosowania nowoczesnych technologii}
\end{table}
\end{frame}

\begin{frame}
 \frametitle{Rozdział 4: Znaczenie technologii informacyjnych w społeczeństwie}
\begin{itemize}
\item
  edukacja zdalna i platformy e-learningowe, umożliwiające naukę na
  odległość, współpracę między uczniami i nauczycielami, tworzenie
  wirtualnych laboratoriów i zasobów edukacyjnych.
\item
  bankowość elektroniczna i płatności mobilne, przyspieszające
  transakcje, zmniejszające ryzyko błędów, oferujące nowe metody
  zarządzania finansami osobistymi i firmowymi.
\item
  systemy zarządzania przedsiębiorstwami (ERP, CRM), wspierające
  planowanie zasobów, analizę danych, obsługę klienta, raportowanie i
  przewidywanie trendów rynkowych.
\item
  telemedycyna i cyfrowe rejestracje pacjentów, pozwalające na zdalne
  konsultacje, monitorowanie stanu zdrowia w czasie rzeczywistym,
  analizę danych medycznych i rozwój inteligentnych systemów
  diagnostycznych.
\end{itemize}
\end{frame}

\begin{frame}
 \frametitle{Rozdział 4: Znaczenie technologii informacyjnych w społeczeństwie}
\begin{itemize}
\item
  administracja elektroniczna, umożliwiająca składanie dokumentów
  online, zarządzanie sprawami urzędowymi, przyspieszanie procesów
  decyzyjnych oraz zwiększanie transparentności działania instytucji
  publicznych. Technologie informacyjne wprowadzają też nowe standardy w
  dziedzinie komunikacji, zarządzania projektami i współpracy
  międzynarodowej. Ułatwiają wymianę wiedzy, rozwój społeczności
  naukowych i branżowych, a także przyczyniają się do szybkiego
  reagowania na kryzysy, takie jak katastrofy naturalne czy zagrożenia
  zdrowotne.
\item
  nauka oraz materiały naukowe, m. in. w matematyce:
\end{itemize}
$x \in (0; 1) \cup (2; 5); \alpha; \Omega; \Leftrightarrow; \int$
\end{frame}

\begin{frame}
 \frametitle{Rozdział 5.1: Rozwój technologii}
 Przyszłość technologii informacyjnych wiąże się z coraz głębszą integracją z codziennym życiem. Jak podaje autor przykładowego artykułu, \cite{artykul} rozwój takich dziedzin jak Internet Rzeczy, sztuczna inteligencja, blockchain, rzeczywistość rozszerzona, komputery kwantowe, automatyzacja przemysłowa i robotyka może całkowicie odmienić sposób, w jaki przetwarzamy dane, podejmujemy decyzje i komunikujemy się.
\end{frame}

\begin{frame}
 \frametitle{Rozdział 5.2: Wyzwania technologiczne przyszłości}
\begin{itemize}
\item
  Prywatność danych -- ochrona informacji osobistych i firmowych w
  świecie cyfrowym, tworzenie regulacji prawnych i standardów ochrony
  danych.
\item
  Bezpieczeństwo cybernetyczne -- zabezpieczenie systemów przed coraz
  bardziej zaawansowanymi cyberatakami, rozwój technologii
  antywłamaniowych i monitoringu sieciowego.
\item
  Etyka technologiczna -- odpowiedzialne wdrażanie sztucznej
  inteligencji, algorytmów decyzyjnych i systemów automatyzujących pracę
  ludzi.
\item
  Zrównoważony rozwój -- minimalizacja wpływu infrastruktury
  informatycznej na środowisko naturalne, projektowanie
  energooszczędnych systemów i recykling elektroniki.
\end{itemize}
\end{frame}

\begin{frame}
 \frametitle{Rozdział 5.2: Wyzwania technologiczne przyszłości}
\begin{itemize}
\item
  Edukacja i adaptacja społeczna -- konieczność uczenia nowych
  kompetencji w świecie dynamicznie zmieniającej się technologii, rozwój
  szkoleń, kursów online i programów certyfikacyjnych.
\item
  Rozwój technologii informacyjnych nie ogranicza się jedynie do sektora
  biznesowego; znajduje zastosowanie w edukacji, medycynie, logistyce,
  administracji publicznej, badaniach naukowych oraz w codziennym życiu
  każdego człowieka. Przewiduje się, że w nadchodzących latach
  technologia informacyjna stanie się jeszcze bardziej nieodłącznym
  elementem codzienności, a jej znaczenie będzie rosło w każdej
  dziedzinie życia społecznego i gospodarczego.
\end{itemize}
\end{frame}

\begin{frame}
 \frametitle{Rozdział 6: Wnioski}
\includegraphics[scale=0.25]{zdj.jpg}
Technologie informacyjne stanowią fundament nowoczesnego społeczeństwa. Dzięki nim możliwy jest szybki przepływ danych, globalna komunikacja, automatyzacja procesów, rozwój gospodarczy i naukowy, a także tworzenie inteligentnych systemów wspierających życie ludzi. Ich dalszy rozwój będzie miał ogromny wpływ na przyszłość, dlatego zrozumienie zasad ich działania i umiejętność wykorzystania technologii informacyjnych to dziś kluczowa kompetencja.
\end{frame}

\begin{frame}
 \frametitle{Podsumowanie}
W miarę postępu technologicznego wzrasta też potrzeba odpowiedzialnego i świadomego podejścia do korzystania z narzędzi cyfrowych, aby zapewnić bezpieczeństwo, efektywność, równowagę społeczną i zrównoważony rozwój w społeczeństwie. Technologie informacyjne będą nadal napędzać innowacje, zmieniać rynek pracy, edukację i życie codzienne, tworząc jednocześnie nowe możliwości i wyzwania dla ludzi i instytucji.
\end{frame}

\begin{frame}
 Dziękuję za uwagę :)\\
 Pracę wykonał: Filip Godlewski 290315
\end{frame}

\begin{frame}[allowframebreaks]
 \frametitle{Bibliografia}
\begin{thebibliography}{9}

\bibitem{artykul}
J. Kowalski,
\textit{Prawa Autorskie w Pracach Naukowych},
http://przyklad.pwr.pl/prawoautorskie, Wrocław 2025; Dostep: 18 listopada 2025.
\end{thebibliography}
\end{frame}

\end{document}